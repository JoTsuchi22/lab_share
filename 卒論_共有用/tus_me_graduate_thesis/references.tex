\chapter{文献}
\begin{list}{}{\setlength{\leftmargin}{2em}\setlength{\itemindent}{-2em}\setlength{\topsep}{0pt}}
 \item T.J.R. Hughes,
       J.A. Cottrell and Y. Bazilevs,
       Isogeometric analysis: CAD,
       finite elements,
       NURBS,
       exact geometry and mesh refinement,
       Computer Methods in Applied Mechanics and Engineering,
       Vol.194,
       No.39-41,
       pp.4135-4195(2005).
\item  J. Fish,
       The s-version of the finite element method,
       Computers $\&$ Structures,
       Vol.43,
       No.3,
       pp.539-547(1992).
\item  J. Fish,
       S. Markolefas,
       Adaptive s-method for linear elastostatics,
       Computer Methods in Applied Mechanics and Engineering,
       Vol.104,
       No.3,
       pp.363-396(1993).
\item  鈴木克幸,
       大坪英臣,
       閔勝載,
       白石卓士郎,
       重合メッシュ法による船体構造のマルチスケール解析,
       日本計算工学会論文集(1999).
\item  中住昭吾,
       鈴木克幸,
       藤井大地,
       大坪英臣,
       重合メッシュ法による穴あき板の解析に関する一考察,
       日本計算工学会論文集(2001).
\item  岡田裕,
       遠藤明香,
       菊池正紀,
       重合メッシュ法による二次元破壊力学解析,
       日本機械学会論文集A編,
       Vol.71,
       No.704,
       pp.677-684(2005).
\item  渡邊梨乃,
       重合パッチ法(S-version Isogeometric Analysis Method, S-IGA)の提案,
       東京理科大学大学院理工学研究科機械工学専攻2020年度修士論文(2021).
\item  J.A. Cottrell, T.J.R. Hughes and Y. Bazilevs,
       Isogeometric Analysis: Toward Integration of CAD and FEA,
       pp.18-68(2009).
\item  L. Piegl,
       W. Tiller,
       The NURBS Book,
       Springer Science $\&$ Business Media,
       pp.188-212(1996).
 \item 長島彩華,
       Isogeometric Analysis(IGA)を用いた二次元線形破壊力学解析に関する研究
       (き裂解析におけるIGA特異パッチの提案),
       東京理科大学大学院理工学研究科機械工学専攻2019年度修士論文(2020).
\end{list}