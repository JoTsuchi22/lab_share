\chapter{重合パッチ法}
\section{重合パッチ法の理論}
重合パッチ法による解析では,
全体の構造をIGAグローバルパッチで離散化し,
詳細な領域ではIGAローカルパッチで離散化し,
グローバルパッチとローカルパッチを重ね合わせて解析を行う.
図~\ref{fig:Concept S-IGA}に示すように,グローバル領域$\Omega^G$,ローカル領域$\Omega^L$,グローバル領域の境界$\Gamma^G$,
ローカル領域の境界$\Gamma^L$,グローバル領域上のローカル領域の境界$\Gamma^{GL}$とする.

\begin{figure}[htbp]
  \centering
  \includegraphics[keepaspectratio, scale = 0.7]
  {fig/重合パッチ_v2.ai}
  \caption{Concept of S-IGA}
  \label{fig:Concept S-IGA}
\end{figure}

\noindent
添え字$G,L$はそれぞれグローバル領域$\Omega^G$,ローカル領域$\Omega^L$に関する物理量量であることを示す表記である.
$\Omega^G,\Omega^L$ではそれぞれ独立の変位場$\boldsymbol{u}^G,\boldsymbol{u}^L$が定義されており,$\Omega^L$上での実際の変位は
$\Omega^G,\Omega^L$での変位を重ね合わせた和で以下のように定義される.

\begin{equation}
  \label{eq:ugul}
  \boldsymbol{u} = \left \{
    \begin{array}{l}
      \boldsymbol{u}^G  \ \ \ \quad \qquad $in$ \quad \Omega^G \cap \overline{\Omega^L}\\
      \boldsymbol{u}^G+\boldsymbol{u}^L \ \ \ \quad $in$ \quad \Omega^L
    \end{array}
  \right.
\end{equation}

\noindent
また,$\Gamma^{GL}$での変位の$C^0$連続性を保証するため以下のような条件を課す.

\begin{equation}
  \boldsymbol{u}^L = \boldsymbol{0} \ \ \ \quad $on$ \quad \Gamma^{GL}
\end{equation}

\noindent
式~(\ref{eq:ugul})を偏微分することで,ひずみは以下のように表される.

\begin{equation}
  \boldsymbol{\varepsilon} = \left \{
    \begin{array}{l}
      \boldsymbol{L}\boldsymbol{u}^G = \boldsymbol{\varepsilon}^G \ \ \ \ \ \  \qquad \qquad \qquad \qquad \qquad \quad $in$ \quad \Omega^G \cap \overline{\Omega^L}\\
      \boldsymbol{L}(\boldsymbol{u}^G+\boldsymbol{u}^L)=\boldsymbol{L}\boldsymbol{u}^G+\boldsymbol{L}\boldsymbol{u}^L=\boldsymbol{\varepsilon}^G+\boldsymbol{\varepsilon}^L \qquad $in$ \quad \Omega^L
    \end{array}
  \right.
\end{equation}

\noindent
ここで,$\boldsymbol{L}$は,式~(\ref{eq:L})で表される二次元の微分作用素である.
領域$\Omega^G$,領域$\Omega^L$において,変位場及びひずみ場は形状関数を用いて離散化され,以下のように表される.

\begin{eqnarray}
  \boldsymbol{u}^G&=\boldsymbol{N}^G\boldsymbol{d}^G\\
  \boldsymbol{u}^L&=\boldsymbol{N}^L\boldsymbol{d}^L\\
  \boldsymbol{\varepsilon}^G&=\boldsymbol{B}^G\boldsymbol{u}^G\\
  \boldsymbol{\varepsilon}^L&=\boldsymbol{B}^L\boldsymbol{u}^L
\end{eqnarray}

\noindent
ここで,$\boldsymbol{N}^G$,$\boldsymbol{N}^L$はNURBS基底関数マトリクス,
$\boldsymbol{d}^G,\boldsymbol{d}^L$はコントロールポイント変位ベクトル,
$\boldsymbol{B}^G$,$\boldsymbol{B}^L$は変位ひずみマトリクスである.
このときグローバルパッチとローカルパッチの
コントロールポイントやノットベクトルが一致する必要はない.

上記の仮定を仮想仕事の原理に代入し,離散化されたつりあい方程式を導く.
ここで,仮想変位$\delta \boldsymbol{u}$,仮想ひずみ$\delta \boldsymbol{\varepsilon}$は,変位,ひずみと同様に
グローバルパッチとローカルパッチの重ね合わせで表現され,以下のように表される.

\begin{align}
  \delta \boldsymbol{u} &= \delta \boldsymbol{u}^G + \delta \boldsymbol{u}^L\\
  \delta \boldsymbol{\varepsilon} &= \delta \boldsymbol{\varepsilon}^G + \delta \boldsymbol{\varepsilon}^L
\end{align}

\noindent
線形弾性体の場合,仮想仕事の原理は以下のように表される.

\begin{equation}
  \label{eq:VW01}
  \int_\Omega \delta \boldsymbol{\varepsilon}^T \boldsymbol{D} \boldsymbol{\varepsilon} d\Omega
  = \int_\Omega \delta \boldsymbol{u}^T \boldsymbol{b} d\Omega + \int_{\Gamma^t} \delta \boldsymbol{u}^T \boldsymbol{t} d\Gamma
\end{equation}

\noindent
ここで,$\boldsymbol{D}$は弾性マトリクス,$\boldsymbol{b}$は体積力ベクトル,
$\boldsymbol{t}$は表面力ベクトルである.
式~(\ref{eq:VW01})の左辺を$\Omega^G\cap\overline{\Omega^L}$と$\Omega^L$の二つの領域に分割すると以下のように表される.

\begin{align}
  \int_{\Omega} \delta\boldsymbol{\varepsilon}^T\boldsymbol{D}\boldsymbol{\varepsilon} d\Omega
  &=\int_{\Omega^G\cap\overline{\Omega^L}} \delta{\boldsymbol{\varepsilon}^G}^T\boldsymbol{D}\boldsymbol{\varepsilon}^G d\Omega
  +\int_{\Omega^L} \delta({\boldsymbol{\varepsilon}^G}^T+{\boldsymbol{\varepsilon}^L}^T)\boldsymbol{D}(\boldsymbol{\varepsilon}^G+\boldsymbol{\varepsilon}^L) d\Omega \nonumber \\
  &=\int_{\Omega^G} \delta{\boldsymbol{\varepsilon}^G}^T\boldsymbol{D}\boldsymbol{\varepsilon}^G d\Omega
  +\int_{\Omega^L} \delta{\boldsymbol{\varepsilon}^G}^T\boldsymbol{D}\boldsymbol{\varepsilon}^L d\Omega
  +\int_{\Omega^L} \delta{\boldsymbol{\varepsilon}^L}^T\boldsymbol{D}\boldsymbol{\varepsilon}^G d\Omega
  +\int_{\Omega^L} \delta{\boldsymbol{\varepsilon}^L}^T\boldsymbol{D}\boldsymbol{\varepsilon}^L d\Omega
\end{align}

\noindent
式~(\ref{eq:VW01})の右辺についても同様に,$\Omega^G\cap\overline{\Omega^L}$と$\Omega^L$の二つの領域に分割すると以下のように表される.

\begin{equation}
  \int_\Omega \delta\boldsymbol{u}^T\boldsymbol{b} d\Omega+\int_{\Gamma^t} \delta\boldsymbol{u}^T\boldsymbol{t} d\Gamma
  =\int_{\Omega^G} \delta{\boldsymbol{u}^G}^T\boldsymbol{b} d\Omega+\int_{\Gamma^t} \delta{\boldsymbol{u}^G}^T\boldsymbol{t} d\Gamma
  +\int_{\Omega^L} \delta{\boldsymbol{u}^L}^T\boldsymbol{b} d\Omega+\int_{\Gamma^t} \delta{\boldsymbol{u}^L}^T\boldsymbol{t} d\Gamma
\end{equation}

\noindent
仮想変位の任意性から,離散化したつりあい方程式は以下のように表される.

\begin{equation}
  \label{eq:K_array}
  \begin{bmatrix}
    \boldsymbol{K}^G & \boldsymbol{K}^{GL} \\
    \boldsymbol{K}^{LG} & \boldsymbol{K}^L
  \end{bmatrix}
  \left\{
  \begin{array}{c}
    \boldsymbol{d}^G \\
    \boldsymbol{d}^L
  \end{array}
  \right\} =
  \left\{
  \begin{array}{c}
    \boldsymbol{f}^G \\
    \boldsymbol{f}^L
  \end{array}
  \right\}
\end{equation}

\noindent
ここで,以下のような式が成り立つ.

\begin{eqnarray}
  \boldsymbol{K}^G&=\int_{\Omega^G} {\boldsymbol{B}^G}^T\boldsymbol{D}\boldsymbol{B}^Gd\Omega\\
  \boldsymbol{K}^L&=\int_{\Omega^L} {\boldsymbol{B}^L}^T\boldsymbol{D}\boldsymbol{B}^Ld\Omega\\
  \boldsymbol{K}^{GL}&=\int_{\Omega^L} {\boldsymbol{B}^G}^T\boldsymbol{D}\boldsymbol{B}^Ld\Omega\\
  \boldsymbol{K}^{LG}&=\int_{\Omega^L} {\boldsymbol{B}^L}^T\boldsymbol{D}\boldsymbol{B}^Gd\Omega\\
  \boldsymbol{f}^{G}&=\int_{\Omega^G} {\boldsymbol{N}^G}^T\boldsymbol{b}d\Omega+\int_{\Gamma^t} {\boldsymbol{N}^G}^T\boldsymbol{t}d\Gamma\\
  \boldsymbol{f}^{L}&=\int_{\Omega^L} {\boldsymbol{N}^L}^T\boldsymbol{b}d\Omega+\int_{\Gamma^t} {\boldsymbol{N}^L}^T\boldsymbol{t}d\Gamma
\end{eqnarray}

\noindent
$\boldsymbol{K}^G$,$\boldsymbol{K}^L$はそれぞれ,グローバルパッチ及びローカルパッチ上で定義される通常の剛性マトリクスである.
$\boldsymbol{K}^{GL}$,$\boldsymbol{K}^{LG}$は両パッチの連成を表すマトリクスであり,結合剛性マトリクスと呼ぶ.
$\boldsymbol{K}^{LG}$はその定義から$\boldsymbol{K}^{GL}$の転置行列となるので,
式~(\ref{eq:K_array})は以下のように書き換えることができる.

\begin{equation}
    \begin{bmatrix}
      \boldsymbol{K}^G & \boldsymbol{K}^{GL} \\
      {\boldsymbol{K}^{GL}}^T & \boldsymbol{K}^L
    \end{bmatrix}
    \left\{
    \begin{array}{c}
      \boldsymbol{d}^G \\
      \boldsymbol{d}^L
    \end{array}
    \right\} =
    \left\{
    \begin{array}{c}
      \boldsymbol{f}^G \\
      \boldsymbol{f}^L
    \end{array}
    \right\}
\end{equation}

\section{結合剛性マトリクス}
結合剛性マトリクス$\boldsymbol{K}^{GL}$の積分計算は
ローカルパッチ上の要素であるローカル要素に対して行う.
ローカル要素の物理座標での数値積分を親要素空間に変換し,
ガウス・ルジャンドル積分を用いて数値積分を行うと以下のように表される.

\begin{align}
  \boldsymbol{K}^{GL}&=\int_{\Omega^L} {\boldsymbol{B}^G}^T\boldsymbol{D}\boldsymbol{B}^Ld\Omega \nonumber \\
  &=\iint_{\Omega^L} {\boldsymbol{B}^G}^T\boldsymbol{D}\boldsymbol{B}^Ldx dy \nonumber \\
  &=\int_{-1}^1\int_{-1}^1 {\boldsymbol{B}^G}^T\boldsymbol{D}\boldsymbol{B}^L|\boldsymbol{J}|d\tilde{\xi} d\tilde{\eta} \nonumber \\
  &=\sum^m_{j=1}\sum^n_{i=1} {\boldsymbol{B}^G}^T\left(\tilde{\xi}_i^G,\tilde{\eta}_j^G\right)\boldsymbol{D}\boldsymbol{B}^L\left(\tilde{\xi}_i^L,\tilde{\eta}_j^L\right)|\boldsymbol{J}|w_iw_j
\end{align}

\noindent
ここで,$n,m,w_i,w_j$はそれぞれガウス・ルジャンドル積分での各方向の積分点数,重みである.
$|\boldsymbol{J}|$はヤコビアンの行列式であり,ヤコビアンの導出を以下に示す.

\begin{equation}
  \boldsymbol{J} = \left[
    \frac{\partial \boldsymbol{x}}{\partial \boldsymbol{\tilde{\xi}}}
  \right] =
  \left[
    \begin{array}{cc}
      \cfrac{\partial x}{\partial \tilde{\xi}} & \cfrac{\partial x}{\partial \tilde{\eta}}\\
      \cfrac{\partial y}{\partial \tilde{\xi}} & \cfrac{\partial y}{\partial \tilde{\eta}}
    \end{array}
  \right] =
  \left[
    \begin{array}{cc}
      \cfrac{\partial x(\xi,\ \eta)}{\partial R(\xi,\ \eta)} \cfrac{\partial R(\xi,\ \eta)}{\partial \xi} \cfrac{\partial \xi}{\partial \tilde{\xi}} & \cfrac{\partial x(\xi,\ \eta)}{\partial R(\xi,\ \eta)} \cfrac{\partial R(\xi,\ \eta)}{\partial \eta} \cfrac{\partial \eta}{\partial \tilde{\eta}}\\
      \cfrac{\partial y(\xi,\ \eta)}{\partial R(\xi,\ \eta)} \cfrac{\partial R(\xi,\ \eta)}{\partial \xi} \cfrac{\partial \xi}{\partial \tilde{\xi}} & \cfrac{\partial y(\xi,\ \eta)}{\partial R(\xi,\ \eta)} \cfrac{\partial R(\xi,\ \eta)}{\partial \eta} \cfrac{\partial \eta}{\partial \tilde{\eta}}
    \end{array}
  \right]
\end{equation}

\noindent
式~(\ref{eq:difinement of u})より,以下の式が得られる.

\begin{align}
  \frac{\partial x(\xi,\ \eta)}{\partial R(\xi,\ \eta)} &= \sum^{n_{en}}_{a = 1} x^e_a\\
  \frac{\partial y(\xi,\ \eta)}{\partial R(\xi,\ \eta)} &= \sum^{n_{en}}_{a = 1} y^e_a\\
  \boldsymbol{X} &\equiv \left[
    \begin{array}{c}
      \boldsymbol{x}^e\\
      \boldsymbol{y}^e
    \end{array}
  \right]^T = \left[
    \begin{array}{cccc}
      x^e_1 & x^e_2 & \cdots & x^e_{n_{en}}\\
      y^e_1 & y^e_2 & \cdots & y^e_{n_{en}}
    \end{array}
  \right]^T
\end{align}

\newpage

\noindent
式~(\ref{eq:d of NURBS shape func})を2パラメータ空間のNURBS曲面に拡張した式を考えると,以下の式が得られる.

\begin{align}
  \frac{\partial R(\xi,\ \eta)}{\partial \xi}  &= \frac{\partial}{\partial \xi}R^{p,q}_a(\xi,\ \eta)\\
  \frac{\partial R(\xi,\ \eta)}{\partial \eta} &= \frac{\partial}{\partial \eta}R^{p,q}_a(\xi,\ \eta)\\
  \boldsymbol{R}'^{a} &\equiv \left\{
    \begin{array}{c}
      \cfrac{\partial}{\partial \xi}R^{p,q}_a(\xi,\ \eta)\\
      \cfrac{\partial}{\partial \eta}R^{p,q}_a(\xi,\ \eta)
    \end{array}
  \right\}
\end{align}

\noindent
また,図~\ref{fig:parent space}に示すような座標変換を考えると式~(\ref{eq:parent01})及び式~(\ref{eq:parent02})より,
以下の式が得られる.

\begin{align}
  \frac{\partial \xi}{\partial \tilde{\xi}} &= \frac{\xi_{i+1} - \xi_i}{2}\\
  \frac{\partial \eta}{\partial \tilde{\eta}} &= \frac{\eta_{j+1} - \eta_j}{2}\\
  \boldsymbol{P}' &\equiv \left\{
    \begin{array}{c}
      \cfrac{\xi_{i+1} - \xi_i}{2}\\
      \cfrac{\eta_{j+1} - \eta_j}{2}
    \end{array}
  \right\}
\end{align}

\noindent
従って,ヤコビアンは式~(\ref{eq:JAC})のように表される.

\begin{equation}
  \label{eq:JAC}
  J_{\hat{i}\hat{j}} = \sum^{n_{en}}_{a=1} X_{a\hat{i}} R'^{a}_{\hat{j}} P'_{\hat{j}}
\end{equation}

$\boldsymbol{B}^G$の計算は,
積分点の座標値としてローカルパッチ上の親要素空間の座標$(\tilde{\xi}_i^L,\tilde{\eta}_j^L)$ではなく,
グローバルパッチ上の親要素空間の座標$(\tilde{\xi}_i^G,\tilde{\eta}_j^G)$を利用するため,
座標変換が必要となる.
ローカルパッチ上の親要素空間からグローバルパッチ上の親要素空間への
座標変換は以下の手順で行う.

\begin{enumerate}
  \item 各積分点のローカルパッチ上の親要素空間の座標$(\tilde{\xi}_i^L,\tilde{\eta}_j^L)$を
        ローカルパッチ上のパラメータ空間の座標$(\xi_i^L,\eta_j^L)$に変換する.
  \item ローカルパッチ上のパラメータ空間の座標$(\xi_i^L,\eta_j^L)$を
        ローカルパッチ上の物理空間の座標$(x,y)$に変換する.
  \item 物理空間の座標$(x,y)$を
        グローバルパッチ上のパラメータ空間の座標$(\xi_i^G,\eta_j^G)$に変換する.
  \item グローバルパッチ上のパラメータ空間の座標$(\xi_i^G,\eta_j^G)$を
        グローバルパッチ上の親要素空間の座標$(\tilde{\xi}_i^G,\tilde{\eta}_j^G)$に変換する.
\end{enumerate}

\begin{figure}[htbp]
  \centering
  \includegraphics[keepaspectratio, scale = 0.85]
  {fig/Coordinate_Transformation.ai}
  \caption{Coordinate Transformation between Global patch and Local patch}
  \label{fig:Coordinate Transformation}
\end{figure}

\noindent
図~\ref{fig:Coordinate Transformation}に示す手順で座標変換を行うが,物理空間の座標をグローバルパッチ上のパラメータ空間の座標に変換する際,
Newtoon-Raphson法を用いる.
まず,物理空間座標を$(x_i,y_j)=(X,Y)$として
以下の方程式を定義する.

\begin{equation}
  \left\{
  \begin{array}{l}
    f(\xi^G,\eta^G)=X-x(\xi^G,\eta^G)=0\\
    g(\xi^G,\eta^G)=Y-y(\xi^G,\eta^G)=0
  \end{array}
  \right.
\end{equation}

\noindent
これらを満たす$(\xi^G,\eta^G)$を反復計算で求めることを考える.
反復回数を$n$とすると,二次元のNewton-Raphson法は以下のように定義できる.

\begin{align}
  \left\{
    \begin{array}{l}
      \xi^G_{n+1}\\
      \eta^G_{n+1}
    \end{array}
  \right\}&=
  \left\{
    \begin{array}{l}
      \xi^G_{n}\\
      \eta^G_{n}
    \end{array}
  \right\}-
  {\left[
    \begin{array}{cc}
      \cfrac{\partial{f(\xi^G,\eta^G)}}{\partial{\xi^G}} & \cfrac{\partial{f(\xi^G,\eta^G)}}{\partial{\eta^G}} \\
      \cfrac{\partial{g(\xi^G,\eta^G)}}{\partial{\xi^G}} & \cfrac{\partial{g(\xi^G,\eta^G)}}{\partial{\eta^G}}
    \end{array}
  \right]}^{-1}
  \left\{
    \begin{array}{l}
      f(\xi^G_n,\eta^G_n)\\
      g(\xi^G_n,\eta^G_n)
    \end{array}
  \right\} \nonumber \\
  &=
  \left\{
    \begin{array}{l}
      \xi^G_{n}\\
      \eta^G_{n}
    \end{array}
  \right\}+
  {\left[
    \begin{array}{cc}
      \cfrac{\partial{x(\xi^G,\eta^G)}}{\partial{\xi^G}} & \cfrac{\partial{x(\xi^G,\eta^G)}}{\partial{\eta^G}} \\
      \cfrac{\partial{y(\xi^G,\eta^G)}}{\partial{\xi^G}} & \cfrac{\partial{y(\xi^G,\eta^G)}}{\partial{\eta^G}}
    \end{array}
  \right]}^{-1}
  \left\{
    \begin{array}{l}
      X-x(\xi^G_n,\eta^G_n)\\
      Y-y(\xi^G_n,\eta^G_n)
    \end{array}
  \right\}
\end{align}

\noindent
$(\xi,\eta)$の初期値$(\xi_0,\eta_0)$は
グローバルパッチ上のノットベクトルの最小値と最大値を用いて
以下のように定義した.

\begin{align}
  \xi_0&=\frac{\xi^G_{\rm{min}}+\xi^G_{\rm{max}}}{2}\\
  \eta_0&=\frac{\eta^G_{\rm{min}}+\eta^G_{\rm{max}}}{2}
\end{align}

\noindent
また,収束判定には以下の条件を用いた.

\begin{equation}
  {\left(X-x(\xi_n,\eta_n)\right)}^2+{\left(Y-y(\xi_n,\eta_n)\right)}^2<\varepsilon
\end{equation}

\noindent
本研究では$\varepsilon=1.0\times10^{-14}$として,
反復計算でグローバルパッチ上のパラメータ空間座標を算出した.