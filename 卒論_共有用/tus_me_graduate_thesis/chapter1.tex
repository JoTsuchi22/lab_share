\chapter{緒言}
\section{背景}
\subsection{アイソジオメトリック解析(Isogeometric Analysis, IGA)}
機械部品や構造を設計する際,CAD(Comuputer Aded Design)を用いて設計モデルを生成し,
その形状を再現した有限要素法(Finite Element Method, FEM)解析モデルを作成し,解析を行うという流れが一般的である.
有限要素法解析では解析モデルを作成する解析の前処理が,解析全体の$80\ \%$を占めていると見積もられている.
有限要素法解析モデル生成を短縮するための研究がメッシュフリー法として既に研究されている.
近年注目されているメッシュフリー法がHughes et al.(2005)によって提案されたアイソジオメトリック解析(Isogeometric Analysis, IGA)である.
IGA解析ではCADで作成した設計モデルのデータを用いて解析モデルを作成することができる.
そのため,解析時間の大幅な短縮が期待できる.

また,有限要素法解析で作成される解析モデルはCADデータの近似であり,曲線形状を厳密に表現できない.
そのため,形状による近似誤差である形状誤差が発生する.
細分化を行うことで形状誤差を小さくすることができるが,
この点において,図~\ref{fig:CAD to IGA}に示すように
IGA解析ではCADと同じ幾何学表現である非一様有理Bスプライン(Non-Uniform Rational B-Spline, NURBS)
を用いているため,メッシュの分割数によらず厳密に形状を再現することができる.

さらに,IGA解析では基底関数の次数を自在に操作することができ,高次の基底関数を用いて解析を行うことによって,解析精度の向上が期待できる.

\begin{figure}[htbp]
  \centering
  \includegraphics[keepaspectratio, scale = 0.55]
  {fig/FEMvsIGA.ai}
  \caption{Conversion from CAD model to FEM and IGA model}
  \label{fig:CAD to IGA}
\end{figure}

\clearpage

\subsection{重合パッチ法(S-version Isogeometric Analysis Method, S-IGA)}
構造物を設計する際,破壊力学解析を行うことは,安全性を確保する上で大変重要なことである.
しかし,自動車や船体等の大規模な構造物に対しては,詳細形状まで再現した解析モデルでの解析は,
要素数が莫大になり解析時間が膨大になってしまうため現実的ではない.
しかし,破壊力学においてき裂や円孔などを有する問題の解析を行うためには詳細形状のモデリングが不可欠となる.
そこで,全体解析を比較的粗いモデルで行い,詳細な情報が必要な部分に関しては詳細形状のモデリングを施したモデルで解析を行うマルチスケール解析が用いられる.
マルチスケール解析では,き裂や円孔と全体の解析モデルを分けて作成することで,問題ごとに解析モデルを再度作成する手間が省かれる.

FEMのマルチスケール解析として,Fish et al.(1992, 1993)によって提案された
重合メッシュ法(S-version Finite Element Method, S-FEM)があり,
グローバルメッシュとローカルメッシュを重ね合わせて同時に解析する手法である.
重合メッシュ法を用いた研究には,高精度なズーミング解析手法として使用した研究(鈴木他, 1999)や
ローカルメッシュによってモデルに局部形状を付与することが可能であることを理論的に示し,
モデリング手法として使用した研究(中住他, 2001)などがある.
重合メッシュ法では,グローバルメッシュとローカルメッシュで節点や境界が一致する必要がないため,非常に柔軟なモデリングが可能となる.
重合メッシュ法を破壊力学に適用した研究(岡田他, 2005)をはじめとして破壊力学への応用も行われている.
また,図~\ref{fig:s-iga abst}に示すように
重合メッシュ法の考え方をIGA解析に応用した手法である重合パッチ法(S-version Isogeometric Analysis Method, S-IGA)が
渡邊他(2021)によって提案されている.

\begin{figure}[htbp]
  \centering
  \includegraphics[keepaspectratio, scale = 0.7]
  {fig/s-iga_abst.ai}
  \caption{Overlaying Global patch and Local patch in S-IGA}
  \label{fig:s-iga abst}
\end{figure}

\section{目的}
本研究では,IGA解析及び重合メッシュ法(S-FEM)の考え方をIGA解析に応用した手法である重合パッチ法(S-IGA)の基底関数の次数を高次化した数値解析例を示し,
解析精度の検証を行う.

\section{概要}
本研究ではまず細分化操作や高次化操作等を実装した重合パッチ法解析のインハウスプログラムに必要なインプットデータの編集を行うためのプログラムを作成し,
重合パッチ法解析のインハウスプログラムを高次の基底関数で動作するように再構築した.
数値解析例では,初めに厳密解を有する内圧を受ける厚肉円筒の問題のIGA解析を行い,2次の基底関数と3次の基底関数の誤差精度を比較し,
作成したプログラムが正常に動作しているか確認した.
その後,遠方で一様引張を受ける円孔を有する平板の解析について重合パッチ法解析を行い,
2次と3次の誤差精度の関係とグローバルパッチとローカルパッチの適切な細分化やサイズ比の検討を行った.

\section{本論文の構成}
本論文の構成を各章ごとに以下に示す.
緒言である本章に続く,第2章アイソジオメトリック解析手法では,
非一様有理Bスプライン(NURBS)の定義と細分化操作及び高次化操作の方法,NURBSを用いたIGA解析の概要,解析モデルについて説明する.
第3章重合パッチ法では,重合パッチ法の基本原理の説明を行う.
第4章数値解析例では,通常のIGA解析と重合パッチ法で高次の基底関数を用いた解析を行い,低次の基底関数を用いた解析と解析精度の比較を行う.
第5章考察では,数値解析例での結果の考察を行う.
第6章結言では,本研究の総括を簡潔に述べる.