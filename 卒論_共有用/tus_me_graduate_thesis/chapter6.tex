\chapter{結言}
本研究ではIGA解析及び重合パッチ法解析の基底関数の高次化を行い,誤差精度の検証を行った.
内圧を受ける厚肉円筒の
IGA解析では2次の基底関数より3次の基底関数を用いた方が同自由度数において精度が高くなり,
良好な解析結果を示すとともに,実装したプログラムが正常に動作していることが確認された.

遠方で一様引張を受ける円孔を有する平板の重合パッチ法解析では
変数が多く,いずれかを固定し,他方を変更することで傾向を検証した.
各パッチの次数の組み合わせによる解析精度は基底関数が
グローバルパッチとローカルパッチで共に3次である場合が最も高精度となった.
グローバルパッチやローカルパッチの分割数を変更する解析では,
ローカルパッチ上の応力分布を比較し,IGA解析同様に
2次の基底関数より3次の基底関数を用いた方が精度が高くなることが確認された.
また,重合パッチ法解析においても実装したプログラムが正常に動作していることが確認された.
グローバルパッチとローカルパッチの分割数や全体のサイズについての関係は,
2次の基底関数のみを用いた場合でも明らかになっておらず,本研究で適切なサイズ比を提案した.
誤差の大きさに最も影響するのがグローバルパッチの要素サイズとローカルパッチの全体サイズとの比であり,
次に影響するのが基底関数の次数の関係となった.

IGA解析では3次の基底関数,重合パッチ法解析では
グローバルパッチの要素サイズとローカルパッチの全体サイズとの比を適切に設定し,
グローバルパッチとローカルパッチのいずれも3次の基底関数を用いると
最も高精度な解が得られると考えられる.